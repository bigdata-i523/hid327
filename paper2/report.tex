\documentclass[sigconf]{acmart}

\input{format/i523}

\begin{document}
\title{Using Big Data to Predict the Impact of Driverless Vehicles on the Unemployment Rate in the US}


\author{Paul Marks}
\affiliation{%
  \institution{Indiana University}
  \streetaddress{Online Student}
  \city{Shepherdsville} 
  \state{Kentucky} 
  \postcode{40165}
}
\email{pcmarks@iu.edu}


\begin{abstract}
A future of driverless cars is coming and with it comes a change in some 
industries.  There will be both positive and negative impacts to jobs in these 
industries.  The most obvious negative impacts are to driver-centric jobs such 
as taxi and truck drivers.  Other industries, such as those directly related to 
producing and maintaining driverless cars, will see an incline in the number of 
jobs available.  Data analytics can and will be used to help predict specific 
impacts and their effect on the overall United States (US) employment rate.
\end{abstract}

\keywords{i523, hid327, self-driving, employment rates, driverless, technology 
and unemployment}

\maketitle

\section{Introduction}

Overall there are an estimated four millions jobs\cite{InsuranceJournal} providing 
direct income to individuals and families which are based on a person driving a 
vehicle.  The vehicle itself may vary, cars, short and long-haul trucks, and 
buses to name a few, but they all require driver today.  That requirement, a 
human driver, will no longer be necessary in the future.  Driverless vehicle 
technology is moving forward at fast pace.  As the technology gets better and 
cheaper its ability to alter the employment scene in the US increases.  ``When 
autonomous vehicle saturation peaks, U.S. drivers could see job losses at a 
rate of 25,000 a month, or 300,000 a year.''\cite{CNBC}

While most people think of negative impacts to employment due to driverless 
cars, less driving based jobs, there will also be positive impacts where jobs 
will be created.  In order for the US to best meet the change in job demand, 
it must first be able to understand the impacts.  Big data analytics can help 
to understand what these impacts may be, who they affect, where the impacts 
will be seen the most, and provide an opportunity for those impacted to 
prepare.  This can be aided by data analytic models which can forecast the 
impact to industries in future years.  The impact can then be leveraged more 
locally by breaking down the numbers state by state and city by city.  Job 
losses will have an economic impact to the local economy while new jobs will 
have a positive impact.  However, these changes may not generally occur in 
the same geographic area.  Job losses and new jobs will also impact different 
demographics of workers.

\section{Job Loses}
\subsection{Passenger Transportation}

One job category which will directly impacted by driverless vehicles is the 
transportation of people from place to place.  Those providing on demand 
services such as taxis, ride hailing services (Uber, Lyft), and personal 
chauffeurs employed over 305,100 people with an average pay of \$24,300 across 
the US in 2016.\cite{BLSOnDemandDrivers}  Another 687,200 people were employed as 
bus drivers with an average pay of \$31,920.\cite{BLSBusDrivers}  Combined this is 
almost one million jobs.  These jobs tend to be more concentrated in larger 
cities.  Once driverless vehicles are approved and people become more accepting 
of them these jobs will begin to disappear.  The adoption rate could be faster 
than anticipated because many people may prefer driverless over a human driver 
for safety reasons.  Not only is it expected that driverless vehicles will be 
much safer than those driven by people, they would never be approved if they 
are not, but some people are leery of getting into a vehicle with someone they 
do not know.  
  
\subsection{Goods Transportation}

Aside from transporting people from place to place, vehicles are used to 
transport goods from place to place.  In 2016 the heavy and tractor-trailer 
transportation industry employed 1,871,700 people directly driving the trucks 
with an average salary of \$41,340.\cite{BLSBigTrucks}  In addition delivery 
truck drivers employed another 1,421,400 people with an average wage of 
\$28,390.\cite{BLSDeliveryDrivers}  

Combining passenger and goods transportation accounted for almost 4.3 million 
jobs in the United States.  The total wages earned by them was almost \$150 
billion.  The average employment in 2016 across all twelve months was 144.3 
million.\cite{BLSUSEmployment}  Comparing the employment numbers shows that 
approximately 2.97 percent of jobs in the United States rely directly on the 
ability of a person to perform the function of driving a vehicle.  These jobs 
will be at an increasing threat from driverless vehicles once they are permitted 
on the roads.  As the jobs are eliminated, so is revenue to federal, state, and 
local governments from the reduction of the \$150B in wages.

\subsection{Non-Driving}

The impact of driverless vehicles does not only affect jobs which require a 
driver.  Managing a team of drivers requires support from recruiting to human 
resources to managers.  Many drivers eat at least one meal on the go so a 
reduction in drivers affects businesses such as restaurants, delis, and coffee 
shops.  The ability of driverless vehicles to avoid accidents better than human 
drivers, some estimates up to 90 percent fewer accidents will mean less need for the 
current 445,000 auto repair jobs.\cite{MakeUseOf}  It has to be assumed that 
driverless vehicles will be programmed to obey all traffic regulations so as 
the percentage of human drivers declines the need for vehicle enforcement and 
related court positions will decrease.  The loss of driver income will decrease 
the amount of money to governments who would have to cut back on the number of 
people they employ.  ``Other peripherally-impacted jobs could include street 
meter maids, parking lot attendants, gas station attendants, rental car 
agencies, and more.''\cite{MakeUseOf}  All of these items will factor into the 
impact of driverless vehicles on employment.  To build a proper analysis of 
true, overall impact of driverless vehicles a model must include all these 
factors or it will not account for employment changes in indirect job markets.

\section{New Opportunities}

Not all news about the anticipated proliferation of driverless vehicles is bad.  
Many industries will expand and new ones will start up.  There are many 
companies investing in the hardware and software needed to make the technology 
viable.  Driverless vehicles themselves will provide an opportunity for small 
businesses and entrepreneurs.  If someone does not have to drive the vehicle 
anymore they will have time to do something else.  Easy answers such as 
watching television or a movie, using their smart phone, or a computer are 
possibilities.  However there are many untapped ideas yet to come for someone 
who sees the inside of a vehicle as an open space to be transformed into 
something different.  It has been estimated that driverless vehicles ``could 
add as much as \$2 trillion to the US economy alone by 2050.''\cite{Wired} 

\section{Predicting the Impacts}

So how can big data analytics help to predict the impact of these changes on 
employment in the United States?  The key will be creating good predictive 
models.  The answer is not as simple as estimating the number of jobs which 
will decrease and comparing it to the number of jobs expected to increase.  
Employment and unemployment rates are much more complex to predict.  ``If a 
firm lays off 1000 workers, only a fraction will enter the ranks of the 
unemployed''.\cite{MacroTheory}  This may seem surprising to some, but it is 
explained by the dynamics which need to be modeled.  One of the premises 
of employment models is that ``workers live forever, spending their lives 
moving between unemployment and employment''.\cite{QuantitativeEconomics}  This 
is not saying that each individual worker lives forever, but that as a whole 
workers need to work and will actively seek to be employed.  

A big data analysis must also take into account the demographics of the 
workers.  4.23 percent of African American workers are employed in a driving 
occupation and 3.25 percent are of Hispanic descent.  The top five states 
which employ drivers are California, Texas, New York, Florida, and Illinois.  
These jobs pay on the average more than non-driving jobs for similarly 
skilled people.\cite{InsuranceJournal}  This means that the jobs being lost are 
not easily replaced with jobs of the same pay.  So even for those who are 
able to find other work it means that their standard of living will be 
reduced.  

The further complicate predicting the impact of driverless vehicles is the 
difference in jobs skills needed.  Many of the jobs lost are lower skill jobs; 
they do not require degrees.  Many of the jobs being generated by the 
driverless car industry will be higher skills such as analyzing all the data 
which will be generated by the vehicles and their passengers.  (Wired)  That 
means the model must take into account ``the match between the searchers' 
characteristics and those of the available jobs''.\cite{MacroTheory}  Another 
issue with the impact on employment is that it will not be homogeneous 
throughout the United States.  Studies show that ``job seekers in depressed 
areas may not be able or willing to relocate to areas with better job 
prospects''\cite{NBERw22672}  This means that jobs lost in one area are not 
filled by those who lost their jobs if the replacement job is not in the 
same area.  

\section{Using the Analysis to Curtail the Impact}

One of the models used in employment calculations is the Aggregate 
Demand/Aggregate Supply model.  Aside from demand and supply it takes into 
account ``a wide array of economic events and policy decisions''.\cite{OpenStax} 
This is one important aspect of any model used to help predict the impact of 
driverless cars on the economy; the fact that policy decisions are part of 
the equation.  Driverless cars will not become a reality without federal, 
state, and local governments passing legislation around their use.  In doing 
so the impact of driverless cars may be lessened or spread out over time.  

The models can also be used to help minimize the impact on unemployment.  
In predicting the impact of the change in technology and what areas will 
be impacted the most, the government will be able to proactively take steps 
to train, retrain, or put other efforts into place to offset the impact.  
By doing this type of analysis proactively the impact of such changes can 
also be predicted.  Providing the right information to people can reduce 
unemployment, shorten the time people are without jobs, and decrease the 
time that jobs go unfilled.   However it takes data analysis to do this.  
Job openings and losses, must be examined with other data sources such 
as ``changes in the distribution of jobs across industries and regions, 
shifts in the demographic characteristics of the work force, and other 
changes in the way labor markets operate''.\cite{JOLTS}

\section{Conclusions}

Driverless vehicles will become a reality in everyday life in the near future.  
The ability to move people and cargo from point to point without the need of a 
human driver will impact the millions of jobs which are based on driving.  On 
the surface this does not seem like an issue which requires too much data 
analysis; it is easy to show that a loss in jobs affects employment rates.  
However when it comes to employment the US cannot afford to be passive.  The 
solution requires input from many data sources covering not only the jobs, 
but geographic information and demographic data on those affected.    

By leveraging big data analytics, models can be created which can predict what 
this impact is, what geographic locations will be impacted, and which 
population demographic will be affected the most. The results of such models 
may be stark and they may be surprising.  What is important is to create data 
models which will be based on factual data so that it can be understood before 
the changes take place.  By doing this, by leveraging data, a potential major 
impact to the United States employments rate can be seen in advance and 
prepared for. 
 
This is the real power of using big data analytics to study the impact 
driverless cars will have on the workforce:  knowing what is coming before 
it happens.  This can allow for plans to be made to minimize the impacts and 
prepare those who could be affected the most.  This can have a positive result 
not only at a national level, but specifically at an individual level for 
workers who will be displaced, but get help in preparing for the change which 
sets them up for another opportunity.  This type of analysis is how advances 
in data technology can help society predict and prepare for change in a 
positive manner.

\begin{acks}

  The author would like to thank Dr. Gregor von Laszewski for his
  support of this topic.  While the impact of driverless cars on the 
  employment rate does not necessarily seem to be a big data issue on
  the surface, it became more and more an issue which big data can 
  help with.  This is one of the many uses of big data which will benefit
  society, the ability to help prepare for future issues.

\end{acks}

\bibliographystyle{ACM-Reference-Format}
\bibliography{report} 


\end{document}
