\documentclass[sigconf]{acmart}

\usepackage{hyperref}

\usepackage{endfloat}
\renewcommand{\efloatseparator}{\mbox{}} % no new page between figures

\usepackage{booktabs} % For formal tables

\settopmatter{printacmref=false} % Removes citation information below abstract
\renewcommand\footnotetextcopyrightpermission[1]{} % removes footnote with conference information in first column
\pagestyle{plain} % removes running headers

\begin{document}
\title{Waste in Healthcare
(Using Big Data to help optimize our Healthcare spend)}


\author{Paul Marks}
\orcid{1234-5678-9012}
\affiliation{%
  \institution{Indiana University}
  \streetaddress{107 S Indiana Ave}
  \city{Bloomington} 
  \state{Indiana} 
  \postcode{47405}
}
\email{pcmarks@iu.edu}


% The default list of authors is too long for headers}
\renewcommand{\shortauthors}{B. Trovato et al.}


\begin{abstract}
The cost of healthcare includes the cost of inefficient services.  While this could include topics such as misdiagnosis, less effective treatment plans, and more efficient use of types of services (emergency room vs. immediate care vs. telemedicine) the purpose of this paper is the cost of Fraud, Waste, and Abuse (FW&A) within the system.  The estimate for what percentage of cost are attributable to FW&A can range from 10-15% (NEED STAT).  The question is “How can we use big data analysis to help minimize these costs and thus optimize the money spent on healthcare?”
\end{abstract}

\keywords{Fraud, Waste, Abuse, Healthcare, Medicare, Medicaid, FWA, FW&A}


\maketitle

\section{Introduction}

The \textit{proceedings} are the records of a
conference. ACM seeks to give these
conference by-products a uniform, high-quality appearance.  To do
this, ACM has some rigid requirements for the format of the
proceedings documents: there is a specified format (balanced double
columns), a specified set of fonts (Arial or Helvetica and Times
Roman) in certain specified sizes, a specified live area, centered on
the page, specified size of margins, specified column width and gutter
size.

\section{The Body of The Paper}

Typically, the body of a paper is organized into a hierarchical
structure, with numbered or unnumbered headings for sections,
subsections, sub-subsections, and even smaller sections.  The command
\texttt{{\char'134}section} that precedes this paragraph is part of
such a hierarchy. \LaTeX\ handles the
numbering and placement of these headings for you, when you use the
appropriate heading commands around the titles of the headings.  If
you want a sub-subsection or smaller part to be unnumbered in your
output, simply append an asterisk to the command name.  Examples of
both numbered and unnumbered headings will appear throughout the
balance of this sample document.

Because the entire article is contained in the \textbf{document}
environment, you can indicate the start of a new paragraph with a
blank line in your input file; that is why this sentence forms a
separate paragraph.

\subsection{Type Changes and {\itshape Special} Characters}

We have already seen several typeface changes in this sample.  You can
indicate italicized words or phrases in your text with the command
\texttt{{\char'134}textit}; emboldening with the command
\texttt{{\char'134}textbf} and typewriter-style (for instance, for
computer code) with \texttt{{\char'134}texttt}.  But remember, you do
not have to indicate typestyle changes when such changes are part of
the \textit{structural} elements of your article; for instance, the
heading of this subsection will be in a sans serif\footnote{Another
  footnote here.  Let's make this a rather long one to see how it
  looks. Footnotes must be avoided.} typeface, but that is handled by
the document class file.  Take care with the use of the curly braces
in typeface changes; they mark the beginning and end of the text that
is to be in the different typeface.

You can use whatever symbols, accented characters, or non-English
characters you need anywhere in your document; you can find a complete
list of what is available in the \textit{\LaTeX\ User's Guide}
\cite{Lamport:LaTeX}.

\subsection{Math Equations}

You may want to display math equations in three distinct styles:
inline, numbered or non-numbered display.  Each of
the three are discussed in the next sections.

\subsubsection{Inline (In-text) Equations}

A formula that appears in the running text is called an
inline or in-text formula.  It is produced by the
\textbf{math} environment, which can be
invoked with the usual \texttt{{\char'134}begin\,\ldots{\char'134}end}
construction or with the short form \texttt{\$\,\ldots\$}. You
can use any of the symbols and structures,
from $\alpha$ to $\omega$, available in
\LaTeX~\cite{Lamport:LaTeX}; this section will simply show a
few examples of in-text equations in context. Notice how
this equation:

\begin{math}
  \lim_{n\rightarrow \infty}x=0
\end{math},

set here in in-line math style, looks slightly different when
set in display style.  (See next section).

\subsubsection{Display Equations}

A numbered display equation---one set off by vertical space from the
text and centered horizontally---is produced by the \textbf{equation}
environment. An unnumbered display equation is produced by the
\textbf{displaymath} environment.

Again, in either environment, you can use any of the symbols
and structures available in \LaTeX\@; this section will just
give a couple of examples of display equations in context.
First, consider the equation, shown as an inline equation above:

\begin{equation}
  \lim_{n\rightarrow \infty}x=0
\end{equation}

Notice how it is formatted somewhat differently in
the \textbf{displaymath}
environment.  Now, we'll enter an unnumbered equation:

\begin{displaymath}
  \sum_{i=0}^{\infty} x + 1
\end{displaymath}

and follow it with another numbered equation:

\begin{equation}
  \sum_{i=0}^{\infty}x_i=\int_{0}^{\pi+2} f
\end{equation}

just to demonstrate \LaTeX's able handling of numbering.

\subsection{Citations}

Citations to articles~\cite{bowman:reasoning, clark:pct, braams:babel,
  herlihy:methodology}, conference proceedings~\cite{clark:pct} or
maybe books \cite{Lamport:LaTeX, salas:calculus} listed in the
Bibliography section of your article will occur throughout the text of
your article.  You should use BibTeX to automatically produce this
bibliography; you simply need to insert one of several citation
commands with a key of the item cited in the proper location in the
\texttt{.tex} file~\cite{Lamport:LaTeX}.  The key is a short reference
you invent to uniquely identify each work; in this sample document,
the key is the first author's surname and a word from the title.  This
identifying key is included with each item in the \texttt{.bib} file
for your article.

The details of the construction of the \texttt{.bib} file are beyond
the scope of this sample document, but more information can be found
in the \textit{Author's Guide}, and exhaustive details in the
\textit{\LaTeX\ User's Guide} by Lamport~\shortcite{Lamport:LaTeX}.

This article shows only the plainest form of the citation command,
using \texttt{{\char'134}cite}.

Some examples.  A paginated journal article \cite{Abril07}, an
enumerated journal article \cite{Cohen07}, a reference to an entire
issue \cite{JCohen96}, a monograph (whole book) \cite{Kosiur01}, a
monograph/whole book in a series (see 2a in spec. document)
\cite{Harel79}, a divisible-book such as an anthology or compilation
\cite{Editor00} followed by the same example, however we only output
the series if the volume number is given \cite{Editor00a} (so
Editor00a's series should NOT be present since it has no vol. no.), a
chapter in a divisible book \cite{Spector90}, a chapter in a divisible
book in a series \cite{Douglass98}, a multi-volume work as book
\cite{Knuth97}, an article in a proceedings (of a conference,
symposium, workshop for example) (paginated proceedings article)
\cite{Andler79}, a proceedings article with all possible elements
\cite{Smith10}, an example of an enumerated proceedings article
\cite{VanGundy07}, an informally published work \cite{Harel78}, a
doctoral dissertation \cite{Clarkson85}, a master's thesis:
\cite{anisi03}, an online document / world wide web resource
\cite{Thornburg01, Ablamowicz07, Poker06}, a video game (Case 1)
\cite{Obama08} and (Case 2) \cite{Novak03} and \cite{Lee05} and (Case
3) a patent \cite{JoeScientist001}, work accepted for publication
\cite{rous08}, 'YYYYb'-test for prolific author \cite{SaeediMEJ10} and
\cite{SaeediJETC10}. Other cites might contain 'duplicate' DOI and
URLs (some SIAM articles) \cite{Kirschmer:2010:AEI:1958016.1958018}.
Boris / Barbara Beeton: multi-volume works as books \cite{MR781536}
and \cite{MR781537}.

A couple of citations with DOIs: \cite{2004:ITE:1009386.1010128,
  Kirschmer:2010:AEI:1958016.1958018}.

Online citations: \cite{TUGInstmem, Thornburg01, CTANacmart}.  

We use jabref to manage all citations. A paper without managing a bib
file will be returned without review. in the bibtex file all urls are
added to rfernces with the {\it url} filed. They are not to be
included in the {\it howpublished} or {\it note} field. 


\subsection{Tables}

Because tables cannot be split across pages, the best placement for
them is typically the top of the page nearest their initial cite.  To
ensure this proper ``floating'' placement of tables, use the
environment \textbf{table} to enclose the table's contents and the
table caption.  The contents of the table itself must go in the
\textbf{tabular} environment, to be aligned properly in rows and
columns, with the desired horizontal and vertical rules.  Again,
detailed instructions on \textbf{tabular} material are found in the
\textit{\LaTeX\ User's Guide}.

Immediately following this sentence is the point at which
Table~\ref{tab:freq} is included in the input file; compare the
placement of the table here with the table in the printed output of
this document.

\begin{table}
  \caption{Frequency of Special Characters}
  \label{tab:freq}
  \begin{tabular}{ccl}
    \toprule
    Non-English or Math&Frequency&Comments\\
    \midrule
    \O & 1 in 1,000& For Swedish names\\
    $\pi$ & 1 in 5& Common in math\\
    \$ & 4 in 5 & Used in business\\
    $\Psi^2_1$ & 1 in 40,000& Unexplained usage\\
  \bottomrule
\end{tabular}
\end{table}

To set a wider table, which takes up the whole width of the page's
live area, use the environment \textbf{table*} to enclose the table's
contents and the table caption.  As with a single-column table, this
wide table will ``float'' to a location deemed more desirable.
Immediately following this sentence is the point at which
Table~\ref{tab:commands} is included in the input file; again, it is
instructive to compare the placement of the table here with the table
in the printed output of this document.


\begin{table*}
  \caption{Some Typical Commands}
  \label{tab:commands}
  \begin{tabular}{ccl}
    \toprule
    Command &A Number & Comments\\
    \midrule
    \texttt{{\char'134}author} & 100& Author \\
    \texttt{{\char'134}table}& 300 & For tables\\
    \texttt{{\char'134}table*}& 400& For wider tables\\
    \bottomrule
  \end{tabular}
\end{table*}
% end the environment with {table*}, NOTE not {table}!

It is strongly recommended to use the package booktabs~\cite{Fear05}
and follow its main principles of typography with respect to tables:

\begin{enumerate}
\item Never, ever use vertical rules.
\item Never use double rules.
\end{enumerate}

It is also a good idea not to overuse horizontal rules.


\subsection{Figures}

Like tables, figures cannot be split across pages; the best placement
for them is typically the top or the bottom of the page nearest their
initial cite.  To ensure this proper ``floating'' placement of
figures, use the environment \textbf{figure} to enclose the figure and
its caption.

This sample document contains examples of \texttt{.eps} files to be
displayable with \LaTeX.  If you work with pdf\LaTeX, use files in the
\texttt{.pdf} format.  Note that most modern \TeX\ systems will convert
\texttt{.eps} to \texttt{.pdf} for you on the fly.  More details on
each of these are found in the \textit{Author's Guide}.

\begin{figure}
\includegraphics{images/fly}
\caption{A sample black and white graphic.}
\end{figure}

\begin{figure}
\includegraphics[height=1in, width=1in]{images/fly}
\caption{A sample black and white graphic
that has been resized with the \texttt{includegraphics} command.}
\end{figure}


As was the case with tables, you may want a figure that spans two
columns.  To do this, and still to ensure proper ``floating''
placement of tables, use the environment \textbf{figure*} to enclose
the figure and its caption.  And don't forget to end the environment
with \textbf{figure*}, not \textbf{figure}!

\begin{figure*}
\includegraphics{images/flies}
\caption{A sample black and white graphic
that needs to span two columns of text.}
\end{figure*}


\begin{figure}
\includegraphics[height=1in, width=1in]{images/rosette}
\caption{A sample black and white graphic that has
been resized with the \texttt{includegraphics} command.}
\end{figure}

\subsection{Theorem-like Constructs}

Other common constructs that may occur in your article are the forms
for logical constructs like theorems, axioms, corollaries and proofs.
ACM uses two types of these constructs:  theorem-like and
definition-like.

Here is a theorem:

\begin{theorem}
  Let $f$ be continuous on $[a,b]$.  If $G$ is
  an antiderivative for $f$ on $[a,b]$, then
  \begin{displaymath}
    \int^b_af(t)\,dt = G(b) - G(a).
  \end{displaymath}
\end{theorem}

Here is a definition:

\begin{definition}
  If $z$ is irrational, then by $e^z$ we mean the
  unique number that has
  logarithm $z$:
  \begin{displaymath}
    \log e^z = z.
  \end{displaymath}
\end{definition}

The pre-defined theorem-like constructs are \textbf{theorem},
\textbf{conjecture}, \textbf{proposition}, \textbf{lemma} and
\textbf{corollary}.  The pre-defined de\-fi\-ni\-ti\-on-like
constructs are \textbf{example} and \textbf{definition}.  You can add
your own constructs using the \textsl{amsthm}
interface~\cite{Amsthm15}.  The styles used in the
\verb|\theoremstyle| command are \textbf{acmplain} and
\textbf{acmdefinition}.

Another construct is \textbf{proof}, for example,

\begin{proof}
  Suppose on the contrary there exists a real number $L$ such that
  \begin{displaymath}
    \lim_{x\rightarrow\infty} \frac{f(x)}{g(x)} = L.
  \end{displaymath}
  Then
  \begin{displaymath}
    l=\lim_{x\rightarrow c} f(x)
    = \lim_{x\rightarrow c}
    \left[ g{x} \cdot \frac{f(x)}{g(x)} \right ]
    = \lim_{x\rightarrow c} g(x) \cdot \lim_{x\rightarrow c}
    \frac{f(x)}{g(x)} = 0\cdot L = 0,
  \end{displaymath}
  which contradicts our assumption that $l\neq 0$.
\end{proof}

\section{Conclusions}

This paragraph will end the body of this sample document.  Remember
that you might still have Acknowledgments or Appendices; brief samples
of these follow.  There is still the Bibliography to deal with; and we
will make a disclaimer about that here: with the exception of the
reference to the \LaTeX\ book, the citations in this paper are to
articles which have nothing to do with the present subject and are
used as examples only.



\appendix

%Appendix A
\section{Headings in Appendices}

The rules about hierarchical headings discussed above for the body of
the article are different in the appendices.  In the \textbf{appendix}
environment, the command \textbf{section} is used to indicate the
start of each Appendix, with alphabetic order designation (i.e., the
first is A, the second B, etc.) and a title (if you include one).  So,
if you need hierarchical structure \textit{within} an Appendix, start
with \textbf{subsection} as the highest level. Here is an outline of
the body of this document in Appendix-appropriate form:

\subsection{Introduction}
\subsection{The Body of the Paper}
\subsubsection{Type Changes and  Special Characters}
\subsubsection{Math Equations}
\paragraph{Inline (In-text) Equations}
\paragraph{Display Equations}
\subsubsection{Citations}
\subsubsection{Tables}
\subsubsection{Figures}
\subsubsection{Theorem-like Constructs}
\subsubsection*{A Caveat for the \TeX\ Expert}
\subsection{Conclusions}
\subsection{References}

Generated by bibtex from your \texttt{.bib} file.  Run latex, then
bibtex, then latex twice (to resolve references) to create the
\texttt{.bbl} file.  Insert that \texttt{.bbl} file into the
\texttt{.tex} source file and comment out the command
\texttt{{\char'134}thebibliography}.

% This next section command marks the start of
% Appendix B, and does not continue the present hierarchy

\section{More Help for the Hardy}

Of course, reading the source code is always useful.  The file
\path{acmart.pdf} contains both the user guide and the commented code.

\begin{acks}

  The authors would like to thank Dr. Yuhua Li for providing the
  matlab code of the \textit{BEPS} method.

  The authors would also like to thank the anonymous referees for
  their valuable comments and helpful suggestions. The work is
  supported by the \grantsponsor{GS501100001809}{National Natural
    Science Foundation of
    China}{http://dx.doi.org/10.13039/501100001809} under Grant
  No.:~\grantnum{GS501100001809}{61273304}
  and~\grantnum[http://www.nnsf.cn/youngscientsts]{GS501100001809}{Young
    Scientsts' Support Program}.

\end{acks}

\bibliographystyle{ACM-Reference-Format}
\bibliography{report} 

\end{document}

