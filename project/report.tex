\documentclass[sigconf]{acmart}

\input{format/i523}

\begin{document}
\title{Using Big Data to Minimize Fraud, Waste, and Abuse (FWA) in United States Healthcare}


\author{Paul Marks}
\affiliation{%
  \institution{Indiana University}
  \streetaddress{Online Student}
  \city{Shepherdsville} 
  \state{Kentucky} 
  \postcode{40165}
}
\email{pcmarks@iu.edu}


\begin{abstract}
Aside from people changing their habits, big data analytics
may hold the best possibility for the improvement of
worldwide health.  It will enable the ability to correctly
diagnose patients more quickly, even when the patients may
not be able to be physically seen by a provider.  It will
be used to create treatment plans specific to not only an
illness, but to the patient's overall health condition and
history, demographics, environment, and access to resources.
While it may not solve the problem of everyone not having
access to the best of care, it can help to make sure everyone
can get the best care possible for them.  This paper explores
the ways in which big data is evolving in the field of
healthcare to make these possibilities become realities and
looks at some of the social concerns which could hold it back.
\end{abstract}

\keywords{i523, hid327, healthcare, patient treatment, 
genomics, diagnosis}


\maketitle

\section{Introduction}
There have been many advances in big data analytics over the last 
several years.  More and more data is able to be processed in a 
shorter amount of time.  There are also many new sources of data.  
Data is not what big data is about though.  It is about taking 
data and turning it into information that can be useful.  The 
application of big data can vary, but very few may be more 
important than the ability to use data for the betterment of 
people's health across the globe.  This is one way in which data 
science can make a substantial contribution to humanity.

Making this a reality is not, nor will be, a simple task.  Health 
data itself requires the proper handling of the information as 
it is very sensitive.  On one hand people have a right to privacy.  
On the other, if data is kept isolated, not combined with records 
from other people, then this limits the ability to gather insight 
and find breakthroughs.  They key is to ensure privacy, but keep 
the integrity and relationships of the data in order to preserve 
privacy while gaining insight.  The insight gained has endless 
possibilities.

One issue facing the medical profession today is a lack of trained 
professionals.  The number of patients per healthcare worker around 
the world can vary from more than six per 1,000 people to less than 
one half per 1,000\cite{WHODensity}.  It is easy to see how this 
one fact greatly impacts the expected lifespan of people.  But 
what if a patient could be examined, diagnosed, and have access to 
a treatment plan without a human doctor needed?  It may sound 
futuristic, but the technology is being implemented today thanks 
in part to data analytics.   

The impact of big data on healthcare doesn't stop there.  The 
cost of treating 5 percent of the most chronic conditions can 
consume up to 50 percent of the money spent on 
healthcare\cite{Optum}.  One reason for this is prevention, 
diagnosis, and treatment plans are not optimized.  There is not 
one way to help patients avoid chronic conditions.  It is based 
on many inputs depending on the person, their environment, and 
other factors.  These same aspects impact the effectiveness of 
treatment plans as well.  One size does not fit all.  Through 
analytics many factors are being analyzed along with the results 
of prior plans to determine which methods would be the most 
effective.  Avoiding a chronic condition not only saves money, 
but extends a patient's life and improves the quality of their 
life.  

The ability to take many factors into account for a patient goes 
well beyond chronic conditions.  Genomic technology is 
progressing which is allowing for a person's individual genome to 
be one of the inputs.  Each person on earth has their own specific 
genome with billions of combinations, some of which directly 
impact their health and susceptibility to illnesses.  Through big 
data analytics, this type of analysis may one day be commonplace 
like taking blood pressure and other vital statics into account.  

The discovery of new drugs and how they can be used to treat 
people is being sped up by the power of big data techniques.  
Drug research requires an immense amount of information to be 
correlated and processed.  Big data is helping to speed this up 
and even helps speed up clinical trials by matching the right set 
of circumstances to provide viable results.  

Progress does not always come without drawbacks, and big data 
analytics in healthcare is no exception.

\section{Handling the Data}

\subsection{Security}
Any use of healthcare data must take into account the ability to 
protect the data.  Therefore a brief understanding of the task must 
be addressed. Healthcare information usually has two forms of 
protected information:  Personally Identifiable Information (PII) 
and Protected Health Information (PHI).  In order to be able to keep 
data with this type of information you must follow very strict rules 
on safeguarding it.  The best known regulations are based on the 
Health Insurance Portability and Accountability Act (HIPAA) of 1996.  
Among the governmental standards to comply with HIPAA are the 
Security Control Assessment\cite{CMS} and Defense Information Systems 
Agency's Security Technical Implementation Guides\cite{DISA}.  These 
types of requirements can be costly and require constant changes to 
remain secure.  

Even with the ability to secure the data properly, any company wishing 
to obtain data must have an approved reason to get the information or 
the approval of the patients involved.  Obtaining approval from each 
patient in a big data application is not practical.  Data is needed 
from too many people to obtain approval for each of them.  A common 
way to handle this is through de-identification.

De-identification is the ability to alter the data in such a way that 
you cannot link health information to a person or identify individuals 
in the data.  However, in order for the data to be useful for analysis 
it cannot be changed randomly so the links between certain data 
elements from record to record is lost.  For instance, a diagnosis of 
a specific cancer in a patient must still be able to be linked to 
treatment data, x-rays, blood tests, etc. from that patient.  In other 
words, de-identification has to be done in such a way that the data 
integrity remains in place, but the individual's identity is protected.  
This can become complicated because data elements such as age, sex, 
and geographical location are important.

Fortunately there are software solutions to assist in the 
de-identification of medical information.  The software is broken into 
two categories:  structured data and free-form text.  
De-identification of structured data is generally easier.  The data 
has a known set of fields of which the ones which can identify a person 
and their health are known.  These fields are added to the software and 
algorithms are run against them.  The resultant data is useful for 
analysis, but the identity of any individual is safe.  This is because 
the algorithm changes data in such a manner that it protects the person 
and the data integrity.  Examples of tools in this arena include PARAT 
from Privacy Analytics, Inc., mu-Argus from the Netherlands national 
statistical agency, Cornell Anonymization Toolkit (CAT), an anonymization 
toolkit from the University of Texas at Dallas, sdcMirco from 
r-project.org\cite{eHealthInfo}.  Commercial tools like Privacy Analytics 
Eclipse claim to de-identify 10 million records per day from a variety 
of sources\cite{PrivacyAnalytics}.

Unstructured data is more complex.  The data which needs to be 
de-identified can be located anywhere within the dataset.  This includes 
the text or metadata attached to images such as x-rays.  Vital clinical, 
diagnosis, treatment, and other medical information is also included 
throughout unstructured data.  Not being able to identify all PHI and 
PII can cause privacy concerns.  Not linking all the correct data 
together reduces data integrity which reduces the usefulness of the 
data being studied.  

Being able to properly de-identify and link unstructured data is 
being studied and refined.  There are challenges for solutions to the 
problem.  Informatics for Integrating Biology and the Bedside\cite{i2b2} 
has held challenges to help further solutions for this problem.  The 
most recent was held in 2014.   Track 1 of this challenge noted that 
``Removing protected health information (PHI) is a critical step in 
making medical records accessible to more people, yet it is a very 
difficult and nuanced''\cite{i2b2}.  The ability to properly de-identify 
the data is rooted in the ability for the software to perform natural 
language processing.  The focus of the challenge was all eighteen HIPAA 
defined PHI types\cite{ScienceDirect}.  While not as mainstream as 
de-identifying structured data, the ability to de-identify unstructured 
data will continue to progress and be solved through commercially 
available products over time.

\subsection{Data Sharing}
There are many sources of healthcare data.  This is a major hurdle as the 
data is in different systems which are governed by different entities and 
used for purposes\cite{Datapine}.  Data is stored in claims systems, 
clinical settings, pharmacies, and others.  It is stored in different 
formats.  These sources may not contain similar key data that allows it 
to be easily brought together.  Individual patients usually have a single 
provider who is their primary insurer.  This data is usually in standard 
formats.  However the same patients may have many providers of care using 
different systems.  While most providers leverage electronic health 
records, these systems can contain many free-form text fields, images, and 
other types of fields. These data sets contain a wealth of information, 
but they are missing data which could be vital such as social, 
environmental, and community data.  Other sources of data which could be 
useful are habits which people store on themselves such as food and 
activity tracking they may enter into any number of online 
applications\cite{HCAPop}.

While more data is being collected, there are still barriers to sharing 
it.  There are the security and privacy concerns discussed earlier, but 
also the costs and who pays for them which must be addressed.  There are 
tools and strategies being worked on in the industry to make sharing data 
across disparate systems possible.  So far a widely adopted solution has 
not emerged\cite{HCATop10}.  Until such time that it does, data analytics 
in healthcare will be hampered.

\section{Big Data in a Clinical Setting}
Being a doctor can be like being a human big data machine at times.  They 
take in many variables, process it against the history of information they 
have, and come to some sort of conclusion.  In many cases there are multiple 
diagnosis that can be made.  In fact sometimes there a lot of diagnosis that 
can be made.  Unfortunately while much of the work is very scientific it 
does not mean that coming to a conclusion is a precise science.  

Different doctors have different backgrounds.  They have seen different 
patients, seen different diseases, studied at different locations, and read 
different literature.  In short, their diagnosis is based off of their 
experiences.  Unfortunately experiences are a form of bias.  It is not that 
someone is doing this on purpose for the betterment or detriment of someone, 
but it is how our brains are wired.  Physicians are not immune to this and 
it can affect the ability to treat all patients and conditions equally or 
appropriately\cite{PMC3797360}.  When set up correctly and fine-tuned over 
time, data analytics can minimize biases.

\subsection{Big Data as a Physician Assistant}
What if each doctor had the collective knowledge of others?  That could make 
for better and more accurate diagnoses around the globe.  A doctor in the 
United States would have the knowledge of thousands of years of alternative 
medicine which may only be taught in schools in the far east.  Not only is 
it possible, but big data is making it happen today through technologies such 
as IBM's Watson.

\subsection{IBM's Watson Health}
One of the challenges facing doctors today is the ability to keep up with 
changes in healthcare.  Even doctors who specialize in a field cannot keep 
up with the amount of information that is being published.  One estimate is 
that 8,000 medical journal articles are published each 
day\cite{HealthCareITNews}.  This makes medicine a good fit for big data.  
Watson Health, IBM's name for their cognitive supercomputer focused on 
healthcare, is able to ingest millions of pages of information in seconds.  
This information becomes part of the core information Watson has at its 
disposal as it assists clinicians by offering recommendations for them to 
consider.  In this way Watson is not the final decision maker, but helps 
doctors be better at what they do\cite{PMC4287097}.    

While Watson is delegated to a physician's assistant currently, it may not 
always be so.  In order to test how accurate it is, IBM tried it on 1,000 
patients.  In this test Watson and the attending physician agreed 99 
percent of the time.  In fact, in 30 percent of the cases Watson offered 
pathways which the physician had not considered.  Armed with information 
like this IBM believes that computer cognitive thinking will be mainstream 
in the next ten years\cite{HealthCareITNews}.  Because of advances in other 
technology areas have been progressing so quickly, it is hard to disagree 
with them.  For instance, computers are now able to instantaneously make 
decisions that seemed unimaginable just a few years ago which as lead to 
the realization of autonomous driving vehicles.  The question may not be 
the technology, but if people will accept a diagnosis from a computer 
program such as Watson.

Watson was also tested to see how examining a patient's entire genome would 
be more beneficial than simply running a panel which focuses on a limited 
number of areas most commonly known to be related to the cancer a patient 
may be experiencing.  While the cost of and speed of sequencing a person's 
genome has been reduced, there is still a lot of work to using this data 
for a specific diagnosis and treatment plan. Both Watson and team from the 
New York Genome Center analyzed a patient's genome.  Each of them was able 
to identify gene mutations which would have pointed to a clinical trial or 
drug which may have been a better match than the treatment the patient 
received.  The difference being that it took the team of physicians 
approximately 160 hours to come to their conclusion.  Watson provided its 
results in 10 minutes\cite{IEEESpectrum}.  While not perfect, Watson adds 
another tool doctors can leverage which would allow them to better diagnose 
and treat patients.

How does Watson do it?  It is actually very similar to how a human doctor 
works.  The patient's symptoms and other information is made available to 
Watson.  From there it deduces the relevant elements and leverages any 
background information it may have such as patient and family history, 
labs, x-rays, and other test results.  It then accesses other sources of 
information it has accumulated over time: treatment guidelines, relevant 
articles and studies, and potentially information from other patients 
similar to this patient.  Watson develops hypotheses and runs them through 
a process to test its hypotheses and provide a confidence score for each.  
Watson then provides its recommended treatment options with its confidence 
rating to the physician\cite{BusInsider}.

One advantage of Watson, or any such system, is that every time it is used 
that patient is getting all of its collective knowledge.  Today when a 
patient see a physician they are diagnosed by that physician and maybe 
one or two other people generally from the same office.  However as Watson 
gets \emph{trained} by specialists in such fields as Oncology, every doctor 
who uses Watson's assistance becomes as or more knowledgeable than the 
collective group.  This means that each doctor is providing top of the 
field care even if they are being seen nowhere near a facility that is 
considered as the best world\cite{FortuneTrans}.  A patient in a country 
not seen as having world-class healthcare can get diagnosed as if they were 
at the Sloan-Kettering Cancer Center.  It also means that a patient who may 
be seeing a specialist in one area may be diagnosed with an ailment outside 
of their field.  This can save time in receiving the appropriate diagnosis 
and subsequent treatment which gives patients the best chance for recovery.

There are obstacles to making Watson available worldwide and that is the 
ability to understand different languages.  Watson knows English, Brazilian 
Portuguese, Japanese, and Spanish and is learning others.  As an example, 
IBM, the Cleveland Clinic, and Mubadala are teaming up and are building a 
hospital in the Middle East.  The Cleveland Clinic is already a user of 
Watson Health and is expected to leverage that in the new facility as many 
chronic conditions in the United States are present in the Middle East as 
well.  To prepare for this, IBM is teaching Watson 
Arabic\cite{FortuneArabic}.  As Watson learns more languages it will be able 
to be leveraged in areas around the world which that language is spoken 
allowing for those populations to advance their healthcare knowledge.

Another advantage that Watson has over human physicians is that it never 
forgets.  Even doctors who try to keep up with changes in healthcare, they 
will never be able to remember information as precisely as Watson.  And 
Watson is also consistent.  A single doctor may be mostly consistent, but 
different doctors will provide different diagnoses given the same input.  
Watson will not unless it is programmed differently or new knowledge is 
ingested which can create a more accurate diagnosis.  It also does not 
have bad days, get tired, and is available 24x7x365.  Watson's incremental 
costs, the cost of using it for one or one million patients, is low.  IBM 
has spent billions on it and is continuing to invest, but those costs will 
be spread out as usage goes up thus making Watson cheaper over 
time\cite{BusInsider}.

\subsection{Implementing Big Data Diagnostic Systems}
Leveraging such technologies can be implemented in various ways.  The 
easiest way is to look at them as another tool in a physicians' tool chest.  
Once fully implemented the inclusion of big data assisted technologies will 
be seamless.  Clinical information is being collected digitally on an 
increasing basis.  As vital signs, x-rays, diagnostic images, lab results, 
and even discussions with the patients are collected digitally they will 
become part of the patient's electronic health record and the overall 
collective knowledge base.  Watson or other software could provide insight 
to the physician.  It may be present a collection of diagnoses scored in 
likelihood based on the evidence collected so far\cite{ForbesTrust}.  It 
could provide recommendations for next steps or information which could 
lead to a more complete recommendation.

The idea behind such a system, Watson or any similar tool, is to make 
physicians better through more accurate diagnosis.  It allows for the use 
of big data without removing the human aspect of medicine.  This will 
help to begin to include the big data and computer health diagnosis to 
patients who would otherwise not be open to it.  For many people their 
relationship with their doctor is personal.  They discuss items with their 
doctor they do not discuss with anyone else.  They may not trust a 
computer with their health\cite{ForbesTrust}.  A non-caring, non-breathing 
inanimate object cannot be trusted with something so human.  In this 
implementation a doctor would still be there providing the personal 
interaction with the patient and thus providing them with the best care 
including the collective knowledge of the system. 

\subsection{Replacing Doctors for Routine Visits}
Having a doctor meet with a patientially initiall may not always be 
required.  The ability for big data to leverage healthcare data could 
lead to helping alleviate the shortage of doctors and nurses in the 
United States and around the world.  Worldwide there is an estimated 
shortage of skilled health professionals of 17.4 million of which 2.6 
million are doctors.  The problem does not get much better over time as 
the estimate for 2030 is over 14 million\cite{WHOGHO}.  It takes a lot 
of time and money for a student to achieve the level of knowledge to 
fill these positions.  Unless the students are already in the pipeline 
then there is not a good response to the problem.  People cannot switch 
careers and be a doctor or a nurse in twelve months or some short time-frame.  

Adding new big data doctors is simple.  It is mostly a hardware problem.  
Buy the right equipment, install the right software, train the staff, and 
Dr. Data can see patients.  Leveraging automated machines to take vital 
signs will free up time for staff\cite{NextWeb} similar to how checking out 
via automated tellers at the grocery store has reduced the number of 
cashiers and baggers needed.  A physical office offering virtual doctor's 
visits could be staffed with people trained on the technology more than 
medical professionals.  They would be there to help make sure that people 
are using the machines correctly and to wipe down equipment after a patient 
has used it.  A nurse would be there in case certain patients are unable to 
use the equipment and their information must be taken manually.  They 
could also be there to take blood samples which would be processed by 
automated machines and included in the patient's profile.  

Automated diagnosis systems are in in use today in a limited basis.  In the 
United Kingdom the National Health Service has approved the use of Your.MD 
(an AI powered mobile app) for diagnosis.  When people are comfortable 
using a technology like this it limits the number of more basic cases a 
doctor has to see and allows them to concentrate on more difficult tasks.  
Another tool, Ada, learns a user's history, provides an assessment, and adds 
an option to contact an actual doctor if needed.  Babylon Health takes it 
one step further by adding following ups with users to see how they are 
doing and can even set up a video consultation with a live general 
practitioner if needed\cite{NextWeb}. 

\section{Limiting Epidemics}
Incorporating big data analytics into the healthcare environment has the 
ability to limit the spread of disease by taking current circumstances 
outside of the immediate patient into account.  In a linked system data 
from other local, regional, national, and global patients can leveraged.  
Are there other patients presenting similar circumstances?  Did the other 
patients provide more details or mention something slightly different?  
Taking this into account may help to diagnose a specific person and to 
identify an outbreak of something.  Is a disease spreading?  Did patients 
come from a similar location such as a building?  By being able to 
correlate this information immediately there is the potential to stop an 
outbreak from spreading thus saving an untold number of patients from 
pain and suffering and saving healthcare dollars by not having to treat 
more patients.  Epidemics have an economic impact at many levels 
including ``the micro (individual and household), meso (establishment, 
village or city) and macro (national and 
international)''\cite{WHOResponse}.

\section{Insurance}
The option of having fully automated doctors' visits could alter the 
insurance market as well.  Health insurance is about numbers.  Actuaries 
spend time estimating the health of the consumers they cover and many 
other factors to determine what premium rates to set\cite{Actuary}.  
Insurers make a profit by taking in more money than the costs to 
administrate the plans and the cost of paying for claims combined.  
To reduce the costs of claims they set predetermined prices for services 
rendered by hospitals, physicians, and sometimes pharmacy companies.  
The lower they can drive the cost of the claims they cover the less 
they charge or the more money they make.  Charging less can result in 
making more money as well as more people may choose to purchase 
coverage from that insurer.

By creating an options for autonomous doctor's visits or telemedicine 
an insurance company could save money.  The more methods can be deployed 
which can reduce overall healthcare costs, the less people will pay.  
There are multiple ways in which this can be included to reduce health 
insurance premiums, a high cost item for most people in the United States 
and other countries.  Insurers can work with healthcare providers who 
leverage this technology to create a reimbursement policy that is less 
for services such as telemedicine\cite{MedicalEconomics}.  They could 
also offer plans to potential customers which require basic treatments 
to take place with autonomous or telemedicine options before they go 
to a doctor's office.  This would offer an economic advantage to people 
which in turn can not only lower costs, but help to increase the 
adoption of new technologies.

Such a system is not for everyone or every condition.  The idea is not 
to replace all doctor's visits, but to allow those who are comfortable 
to take advantage of lower cost coverage.  It will encourage younger 
people to keep insurance if it is made more affordable.  Currently the 
highest rate of not having insurance in the United States is when 
someone can no longer be covered as part of their parents plan, starting 
around the age of 25\cite{Census}. 

\section{Portability}
More importantly than lowering the cost of healthcare or making seeing 
a doctor more convenient is the ability to make exceptional healthcare 
available almost anywhere.  Big data using an automated doctor can have 
an impact on under-served areas the like of which no one has ever seen.  
Today there are people who do not have access to healthcare of any kind.  
When they get sick they may not have a place to turn.  In developed 
countries the number of patients per doctor is generally in the low 
hundreds.  In poor, \emph{third world countries} the number of patients 
per doctor is in the thousands or tens of thousands\cite{BigThink}.  
There are people who try to help, such as Doctors Without Borders, by 
making visits to these areas to provide some support but it does not 
reach a level anywhere near what people in some countries have available 
to them.  If each doctor could multiply their impact with technology 
then the under-served would be helped more.  As technology advances so 
people could be seen by experts without one being physically present 
then even more people could be seen.

\section{Patient Data Collection}
\subsection{Actual Data vs. Circumstantial Insight}
The more valid data which can be collected on patients the better big 
data will be able to help improve treatment for people around the world.  
The more accurate the data, the more accurate the analysis and results 
will be.  Fortunately technology is helping in this area as well.  
Many people around the world have access to devices which monitor 
different aspects of our daily lives.  Hundreds of millions of people 
around the world have purchased wearable devices, many of which can be 
used to monitor activity and inactivity\cite{StatistaWearable}.  By the 
end of next year it is expected that over one-third of people in the 
world will own a smart phone which can also track this type of 
activity\cite{StatistaPhones}.  While they are not seen as a medical 
device, they can help to track activity which is useful for diagnosis 
and treatment.  They are another input into the data about a patient 
which can be used to more accurately gather information.  Today doctors 
rely on a patient to answer questions about their level of activity.  
With such a devices they can get a more accurate picture.  

These devices are useful for more than just activity levels.  They 
also provide insight into areas of people's lives they are not really 
able to answer accurately such as how they sleep.  Many people may 
sleep they sleep well or not so well, but in fact they are basing this 
more on how they feel than how much rest and how good of rest they 
get.  Activity trackers are able to track sleep patterns as well.  They 
actively monitor your inactivity.  When used correctly a wearer pushes 
a button to indicate they are going to sleep and when they get up in 
the morning.  The monitor is then able to track how long it takes for 
someone to get into a motionless/restful state.  It continues to 
track them throughout the night recording if they move around, get up, 
etc.  Getting good sleep is a key element of maintaining overall 
health\cite{Health}.

More advanced features of activity trackers include the ability to 
monitor vital signs like heart rates.  They can be extremely important 
to a diagnosis providing input similar to a mini stress test.  This 
is especially true if a person exercises, such as during jogging.  The 
device can monitor how far a person is moving and their associated 
heart-rate.  By gathering this information, the data can be fed into 
patient's profile when they visit a doctor (virtually or physically) 
instead of having to wait for a patient to get a test done and receive 
that feedback.  Shortening the time to collect data and accurately 
analyze the patient can be the difference between life and death.

One aspect of activity trackers which must be noted is their accuracy 
and consistency.  This is something big data can help with as well.   
Steps from person to person are not of consistent stride, tracker 
accuracy changes from device to device, heart rate monitors vary, and 
sleep are not be tracked similarly across all products and types of 
activities\cite{AceFitness}.  Big data can help normalize this input 
so that it can become a reliable input.  Analysis has been done on 
different monitors to see how accurate they are.  In order to bring 
them into health analysis more tests can be performed to get an 
accurate picture of how the devices correlate to the actual distances 
walked and level of sleep.

Activity trackers are only the beginning.  \emph{Wearable technology} 
is an expanding field which is enhancing the collection of passive 
data.  Sensors are being built into clothing which track more accurately 
and include more types of data\cite{DigitalTrends}.  This includes 
information like breathing rate and muscle activity.  They not only 
collect more types of data, but can wirelessly transmit the data via 
Bluetooth\cite{PocketLint}.  This means they can create a more accurate 
picture based on electronic data which can be used as an input.  The 
more this type of technology becomes commonplace, the more data which 
can be fed into a patient's health record and the collection of health 
information.  

\subsection{Follow-Up Visits}
All of these devices also have the ability to not only be used in 
diagnosis, but in the monitoring of treatment plans.  Is the patient 
exercising as they say they are?  Is a medicine or other corrective 
action helping them to lower their heart rate or get more restful 
sleep?  It can also help to notify the patient or doctor when they 
are exceeding a prescribed level of respiration or heart rate.  This 
can trigger an alert for a patient if they are at risk or even that 
they may need to seek treatment.  These levels will not only be set 
based on standards, but patient specific information\cite{PubHealth}.  
They can also take into account the environment the person is in.  
Are they in a hot location or one with high allergy levels which 
could negatively impact them?  This is what separates the treatment 
plans of today with those of tomorrow.  Use the technology to more 
accurately collect data on the patient, use it to create a diagnosis, 
monitor the patient using the technology, feed that data back into 
the patient's health record, and adjust as needed based on factual 
information.

Beyond the use of commercially available monitoring systems, there 
are devices which collect data similar to the information collected 
by a physician.  Simple systems such as a blood pressure monitors 
are common.  Many other pieces of equipment can be prescribed by a 
physician for home monitoring.  These systems not only collect 
information, but are able to digitally transmit the data so that it 
can be automatically analyzed with other sources of information.  
A patient will get feedback without having to visit a 
doctor\cite{PubHealth}.  This helps to close another gap in 
healthcare which affect many people:  not following up with their 
doctor.  Missing these visits can negatively impact the patient.  
By easing the ability to be monitored, automating the data collection, 
and instantly analyzing that data will lead to better overall 
prognosis.

Big data will also help to change people's habits.  By using the 
data collected a picture of potential outcomes can be made for a 
patient to contemplate.  Instead of generalities, patients will 
receive advice based on their medical history, other patients like 
them, treatment plans, and other inputs based on the variables 
specific to the patient's circumstances.  It can show a patient how 
they impact their recovery based on what they are doing or not 
doing.  For instance if they miss taking their medicines on time, 
do not lose weight, continue to smoke, or whatever other variables 
they are in control of and how it affects their specific recovery 
or health status.  Showing them in advance may give them the 
motivation they need to follow the plan more closely.  Throughout 
their treatment the model can be updated based on the patient’s 
actual adherence to the plan.  This provides another feedback loop 
for the patient to course correct their habits if they have not 
been following it as outlined\cite{PubHealth}.

Not only will big data help to diagnosis patients more accurately, 
but it will also allow for the customization of treatment plans at 
levels not available today.  Instead of relying on more general 
treatment plans, patients will have their plans customized by their 
specific set of circumstances.  Demographic information about the 
patient will be used to compare to historical plans and outcomes 
of patients most closely related to their characteristics.  This 
includes not only the patients themselves, but the environments 
they live in.  Pollution, weather, access to ongoing care, income 
(the patient may have to work whereas a long period of rest would 
be better) and other circumstances will be variables which may not 
be controllable by the patient, but can be used to help treat them.  
The plan will not necessarily be the best treatment course, not 
everyone has the access to the best care or the ability to abide 
by it, but will instead be the best plan for them and their 
circumstances.  Each patient will be able to maximize their chances 
of recovering or otherwise leading the most normal life possible. 

\section{Access to Healthcare}
It is estimated that over 400 million people do not have access to 
basic healthcare around the world and others are forced into extreme 
poverty because of what they pay for healthcare\cite{Who400}.  
Through tools referred to as telemedicine, these numbers can be 
lowered.  Telemedicine itself is the ability for people to get 
evaluated, diagnosed, and treated while the physician is not located 
where they are.  When combined with a mobile diagnostic unit a 
patient can get similar care to someone who is seen at a 
clinic\cite{WhoRemote}.  As advances in automated solutions such as 
IBM Watson evolve, there could be a day when these remote services 
are performed in very remote areas where communication with a 
physician would be technically challenging.

\section{Cost Savings}
Another reason why big data will be helping with healthcare more and 
more in the future is the most basic of reasons:  Economics.  
Regardless of the country or political system, there is always an 
economic element which must be addressed.  No country, no system has 
an endless supply of any services or funds.   Because of that ideas 
which make the most economic sense have a better chance to be 
adopted.  The economics of automating healthcare with big data 
analytics will reach a tipping point as time progresses.

Simply put, healthcare is getting more and more expensive every year 
and computing resources become cheaper every year.  Worldwide the per 
capita expense of healthcare has risen from \$661 to \$1,059 (numbers 
in Unites States Dollars or USD) in the last 10 
years\cite{WoldBankPerCapita}. That is a 60.21\% increase in one 
decade.  The average per capita may seem low to some but that is due 
to it being worldwide number.  Many countries spend almost nothing 
on healthcare per capita while others spend thousands.  For instance, 
in 2004 Vietnam spent \$30 USD per capita and \$142 USD in 2014.  
This is a 373\% increase, but in total dollars it is still a fraction 
of \$6,369 (2004) and \$9,403 spent in the United 
States\cite{WoldBankPerCapita}.  

In contrast to this the cost of computing power has decreased year 
over year.  Computer power is not as straightforward to analyze, but 
cost trends are easily seen.  One way is to compare the cost using 
a baseline year and showing other years as a percentage of the cost 
of the baseline.  Using December of 1997 as a baseline (100) of cost 
for computers, the cost of computers and peripherals in January 2004 
had dropped to 16.2.  In other words, to get the same amount of 
computer power in 2004 you only had to spend 16.2 cents for every 
dollar spent in December of 1997.  By January of 2014 it had dropped 
to 4.9.  Comparing the 2004 and 2014 numbers, the same ones used above 
for healthcare spending, the cost of computing had been reduced by 
69.75\%\cite{CompPrices}.  

A specific component when it comes to big data is the cost of storage.  
The decline in the cost of storage over time is staggering.  In the 
early 1980's the cost of on gigabyte (GB) of storage was in the hundreds 
of thousands of dollars.  Using early 2004 as our baseline the cost for 
one GB of storage had dropped to just under \$2.00.  By 2014 the cost 
had declined further to between three and four cents per GB\cite{MKomo}.  
The speed at which the data can now be retrieved as compared to 2004 
is like comparing the speed of light to the speed of sound.  Today's 
storage units are that much faster.

Using this data one can see that as we are able to leverage big data 
solutions to provide better healthcare we can also begin to slow the 
incline of healthcare costs and then lower the cost of healthcare 
over time.  Adding a new virtual doctor will not take years of 
schooling which can cost hundreds of thousands of dollars in some 
countries.  It will be the cost of some piece of common technology 
and a licensing fee for the software.  As with most everything 
technology based, increasing the volume decreases the cost.  So as 
more and more virtual doctors are brought online the cost of each 
will decrease.  

\section{Chronic Conditions}
Chronic conditions are ones that ``are preventable, and frequently 
manageable through early detection, improved diet, exercise, and 
treatment therapy''\cite{FightChronicDisease}.  They are also very 
expensive to manage and treat.  Worldwide in 2010 the total cost of 
heart disease alone was \$863 billion dollars (USD) and is expected 
to be \$1.44 trillion by 2030.  Between 2011 and 2031 the cost of the 
top five chronic diseases (cancer, diabetes, mental illness, heart 
disease, and respiratory disease) will cost \$47 trillion (USD) 
globally\cite{Reuters}.  

It is not only the economic impact of chronic diseases that make them 
a target for big data analysis.  Chronic diseases reduce people's 
quality of life.  This cannot be factored into simple terms such as 
money.  Chronic diseases are the cause of 60 percent of deaths 
worldwide\cite{WHOChronicDisease}.  In a 2002 study it was estimated 
that 84 percent of deaths were due to chronic diseases in Europe and 
Central Asia\cite{PRB}.  Chronic disease is so prevalent and impactful 
to people's lives that it has been labeled as ``the most expensive, 
fastest growing, and most intricate problem facing healthcare providers 
in every nation on earth\cite{HITAnalyticsHow}.''  With data like this 
it is easy to see why advances in chronic diseases is important.  The 
question becomes how do fight them.  

\subsection{Prevention}
The best way to fight chronic disease is to never have one in the first 
place.  The best way to reduce the number of people who get a chronic 
condition is early intervention.  Big data analytics can be used to help 
with population health management when it comes to chronic diseases.  
That is by identifying those who are at a high risk of getting one of 
these costly, harmful conditions\cite{HITAnalyticsHow}.  The ability 
to leverage big data in prevention is a two part process.  First risk 
factors which are modifiable must be identified and then interventions 
need to be created which will have an impact on changing the 
factors\cite{Liebertpub}.  

Modifiable is the key word in the first aspect of using big data.  A key 
to fighting many chronic conditions is for people to stop behaviors such 
as smoking, to eat healthier, and to exercise more.  However, if it was 
as easy as letting people know this then there would be a lot less chronic 
disease already.  Big data can take many factors into account and help 
to create a more precise message for a people with specific risk 
elements.  For instance instead of telling a patient to eat more 
nutritious foods, by leveraging elements of their specific health factors 
a doctor can recommend more precise information such as asking them to 
include a particular dietary nutrient\cite{Liebertpub}.  Big data can also 
help with the timing of the message.  In a survey patients wanted more 
information from analytics that would have warned them before they 
developed a chronic condition\cite{HITAnalyticsChronic}.  When someone 
is presented with more personalized information (they are on a path and 
about to reach a point of no return) vs. general (a healthy lifestyle 
may prevent you having issues years down the road) they are more 
compelled to heed that information and act upon it.  

Newer technologies outside of a clinical setting are helping to add to 
the data available to analyze and care for patients.  Combining data 
from a patient's activity monitor, fitness tracking website, or food logs 
into their plan helps to create a feedback cycle for the healthcare 
provider.  Many applications track food by scanning the USB code from 
the package.  Making it simple helps to get people to do things.  The 
easier it is, the more likely they are to do it.  Taking this data and 
combining it with clinical data such as blood labs and vital statistics 
can show a patient how they are directly impacting their health in a 
positive or negative manner.  It changes the conversation from more of 
a public service announcement general message to one unique to them.

A special sub-section of patients are very high-cost patients.  In the 
United States there are roughly five percent of patients who account 
for almost 50 percent of healthcare spending\cite{HealthAffairs}.  
Identifying these patients and creating intervention plans that work 
can have an enormous impact on their lives and the cost of healthcare 
overall.  Patients with seemingly similar risk factors may have very 
different prognoses.  Obvious factors such as age, weight, sex, and 
vital statistics may be the same.  In order for big data to help 
identify the five percent more data is needed.  Including mental health 
data, genetic information, socioeconomic, martial status, living 
conditions, and even cultural factors into the analysis will allow for 
better predictions and better ways to intervene which will lead to 
better outcomes\cite{HealthAffairs}. 

\subsection{Management}
Even with the best of preventive measures there will still be too many 
people with chronic conditions for years and decades to come.  
Approximately 25 percent of people with chronic conditions have 
restrictions in what tasks they can perform for themselves, at work, 
or at school\cite{ScalableHealth}.  Because of this big data must also 
be leveraged to help manage those with chronic conditions.  Managing 
it is not only based on cost, but helping them to live a better quality 
of life with less trips to the doctors and less admissions to a 
hospital.  Data analytics can help to customize treatment plans to the 
circumstances of each patient.  It can see patterns in patient's data 
and help to determine better follow up schedules.  This could mean the 
difference between a visit with their doctor or a costly 
hospitalization\cite{MedCity}.

Part of the solution for using big data to help tackle chronic 
conditions is leveraging new sources of information from technologies 
such as wearables.  As mentioned earlier they allow for real-time 
data to be collected, combined with other sources of information 
including that of other patients, and provide better treatment plans 
for patients.  Historically the medical profession had to rely on 
subjective input from patients when they came in for a visit.  How 
often were they active, did they log information like their heart 
rate and blood pressure when they should have.  With some wearables 
all this information and more is gathered in real-time and can 
trigger an alert to a care management 
professional\cite{ScalableHealth}.  This means that changes can be 
made when they are needed and the patient can get immediate 
attention, not days or weeks later.

Another issue with chronic care for providers is that patients may 
have multiple conditions.  They may be overweight, have diabetes, 
and hypertension.  This leads a patient to having multiple doctors 
each working on a specific condition, but no real coordination 
across the diseases.  A treatment for one condition may have a 
negative impact on the patient because of treatment or drugs 
prescribed for another condition.  And this situation is not unique 
as there a many patients suffering from the same conditions 
simultaneously.  Big data analytics can bridge this gap.  By combining 
data from multiple sources, patients, and treatments physicians can 
create a customized treatment plan for a patient to combat all three 
illnesses in the best manner without adverse 
interactions\cite{HealthCatalyst}.

The result of this is that big data can help people see that 
treatments are tailored to them and are making a difference.  Data 
analytics allows for patient-centric care, not disease-centric care.  
Patient managers would work with patients providing details on their 
plan, their results, and will be able to show patients how the care 
plan affects their quality of life.  It can help to create a 
healthcare environment ``where patients are not only engaged in time 
but see improved health results at affordable costs''\cite{Innovaccer}.

\section{Genomics (Personalized Healthcare)}
The field of Genomics is investigating how healthcare can be more 
personal.  How diagnosis and treatment plans will be based on a 
specific person instead of how the factors or ailment is normally 
seen and treated in the general population.  This is essential work 
because in the United States up to 47 percent of the cost of 
healthcare is spent on interventions that do not provide any value.  
While the actual percentage may vary in other countries, this is a 
worldwide problem\cite{PMC4287097}.  Any easy way to understand 
the difference is over the counter medicine.  Generally speaking 
the instructions on a bottle are broken down into children and 
adults.  Following the directions adults will take the same amount 
of medicine regardless of their age, weight, or overall health. 

Genomics aims to make medicine very specific to an individual by 
breaking down each person's genome.  This is only possible through 
big data as a single person's genome produces a lot of data because 
it has up to 25,000 genes with three million base pairs.  One human 
genome can produce up to 100 gigabytes of data\cite{OReilly}.  And 
the information from one individual is not what is required for 
personalized health.  It requires genomes from many individuals.  
The more data available, the better the analysis can be on 
similarities between people and how they may react to certain 
treatments.  This multiplies 100 gigabytes by thousands, then 
millions, then hundreds of millions.

Through advances in technology such analysis is possible.  In 2003 
the first human genome was sequenced.  It was only after 13 years 
and approximately \$3 billion dollars.  By 2015 the same work can 
be done in a few hours at a cost of just over 
\$1,000\cite{HITAnalyticsGenomics}.  This means that more and more 
people can have their genomes sequenced and used for analysis and 
personalized diagnosis and treatment of diseases.  As more and 
more genomes are collected and analyzed treatment can be based on 
their personal genome and their family traits through family 
based analysis.  This analysis lets doctors see how people may 
have inherited a propensity to be susceptible to certain diseases 
based on mutations in their genomes.  In addition, through 
population based analysis environmental and cultural factors can 
be included.  It is estimated that by 2025 over 100 million genomes 
could be sequenced\cite{PMC5343946}.  Analyzing the details of 
the building blocks of so many individuals will be an a big data 
challenge which can have an enormous impact on healthcare.

\section{Drug Discovery}
Discovering new drugs which can help us live a better life is 
something like finding a needle in a haystack.  Large libraries 
of molecules have to be examined ``against millions of data points 
spanning chemical, biological, and clinical 
databases''\cite{MITSloan}.  This is done looking for relationships 
between diseases and drugs to see if a particular drug could be 
used to treat the disease.  While the process is not new, this work 
is the basis of many new drug discoveries, the ability of current 
big data techniques speeds up the process allowing for drugs to be 
discovered more quickly\cite{MITSloan}.

One of the reasons for it being so complicated goes back to the 
discussion of the human genome:  each person is a unique individual.  
If you have seen a commercial or advertisement for a prescription 
drug there is always a list, sometimes a very long list, of possible 
side effects.  These are adverse impacts which can range from minor 
annoyances to death.  Part of the challenge of drug discovery is 
attempting to identify and quantify the impact a drug may have on 
people.  To speed this process healthcare big data has developed 
solutions such as array-based technologies which are purpose built 
to combinatorial problems.  This lets researchers find patterns in 
the data more quickly, speeding up the overall 
process\cite{ClinicalLeader}.

Once a drug is thought to have a potential positive use it must go 
through a testing phase before it is approved for use.  This can be 
long process which has successes and failures.  Big data is being 
used for ``the improvement of clinical trial designs (e.g., endpoints, 
inclusion/exclusion criteria, etc.)''\cite{GeneticEpid}.  This not 
only allows for potentially a quicker time to market, and thus the 
ability help people sooner, but a cost savings without paying for 
trials which do not produce viable results.

\section{Incentives for Adoption}
In the end many of the advances will only be possible if people accept 
them.  So how can this number be influenced?  The most logical way to 
do so is to make the adoption of these advances financially 
beneficial.  People are more willing to take a chance when they can 
see a hard benefit.  Insurance premiums can help to drive this and 
provide an immediate benefit.  Plans could be offered in which a 
person's primary care is provided by a big data doctor.  People would 
have to consent to having their information stored electronically and 
compared against the data sets.  Visits to physical doctors including 
for second opinions would be limited.  They could even have different 
reimbursement models similar to preventive tests.  Most insurance 
today covers preventive services at 100 percent and are not subject 
to a deductible.  Electronic visits could be treated similarly.  They 
could be covered at 100 percent, or some number higher than regular 
doctors visits, and may or may not be subject to a deductible.  
Leveraging these types of incentives will help to promote the use of 
advanced analytics in the healthcare field.  As usage grows so will 
the basis of data available to analyze and the ability to create 
better analysis models.

Another incentive for leveraging big data analytics by physicians is 
being led by the governments and private insurance.  Instead of paying 
for services as they are performed, alternate payment models are 
being explored.  For instance, in the United States the Centers for 
Medicaid and Medicare services is creating Alternate Payment Models 
to stimulate high-quality, cost-efficient care\cite{CMSAPM}.  
Physicians are able to earn more income and profit by achieving 
better outcomes.  They will be willing to invest in computer analysis 
which will help them to diagnose and treat patients better.  The 
financial incentive will drive change in providers' habits which 
will benefit the healthcare big data analytics and patients.  

\section{Drawbacks}
Leveraging big data innovations does not come without hurdles.  One 
of the first is that people are generally slow or not open to change.  
The more personal the need for change, the less open they are.  
Organizations (hospitals, physician groups) are no different.  Part of 
being an individual is making choices based of what information you can 
gather and leveraging your ability to make a determination.  This is 
part of what makes each person unique.  It is also how we learn.  The 
more we become dependent on machines, the less we store in our own 
brains and we stop ``building the networks in our brains to solve a 
whole host of problems.\cite{PsychologyToday}''  As those in the 
healthcare field rely more on technology to diagnosis and treat patients, 
the less human innovation may leveraged which can have a detrimental 
effect over time.  

A major complication in big data analytics in any setting is the quality 
of data.  The term emphasis garbage in, garbage out has probably been 
applied to computer systems since the beginning.  There are techniques 
used to combat this, but when it comes to people's health it is a bit 
more important.  A portion of the healthcare data used as a base for 
analysis comes from existing diagnosis and treatment performed by humans.  
In looking at second opinions for patients, it was estimated that 
``10\% to 62\% of second opinions yield a major change in the diagnosis, 
treatment, or prognosis''\cite{MayoClinic}.  Extrapolating this number 
to the base of information in big data for analysis means that a 
significant portion of the data would be different if a patient simply 
went to a different doctor.

Aside from the data itself, there is the potential for the algorithms 
behind big data analysis to be biased or having discrimination built 
into them.  There has been a lot of talk about a lack of diversity in 
the technology world, especially with companies in Silicon Valley.  
This lack of diversity could become manifested into the analytics behind 
healthcare analytics.  Different cultures and different races have some 
unique healthcare challenges.  With a lack of diversification in key 
jobs the developers of healthcare systems could under-serve large 
portions of the world's population due to a lack of understanding of 
how certain diseases affect their everyday lives.  The United States 
Federal Trade Commission has asked companies in general to look at 
how representative their big data is and whether their models have 
built in biases\cite{TheRegister}.  The fact that healthcare around 
the world varies based economic factors makes it easy to understand 
how the data itself can be discriminatory.  More wealthy people will 
be proportionally more represented than the poor thus skewing the data 
toward conditions afflicting the wealthy.  

While big data will help to diagnose patients and create treatment 
plans, it does not come without it drawbacks.  One of the biggest may 
be innovation.  Part of being human is the ability to think of what 
has not already been done before.  As algorithms and data analysis 
based on the historical variables begin to become more commonplace, 
there will be a reduction in the human factor of the medical 
profession.  When faced with what can seem like a dire situation, the 
human mind can think of new options not previously discovered.  
Trying something which may not seem to have an impact on the surface, 
but something completely unrelated to any prior decision made can lead 
to new alternatives.  What will a computer do with a patient when it 
does not see any hope?  A human physician may opt to take a risk.  
It is a well-informed risk with the patient knowing that there are no 
guarantees.  It is easy to assume when an automated course of action 
without a substantial chance of a positive outcome is encountered that 
a physician would be able to intervene.  This is true for a while, 
but as more and more of medicine is turned over to computer diagnosis 
and treatments the pool of capable physicians will shrink.  With less 
people involved the less chance there is that the truly gifted 
individuals who make strides in the field will even decide to enter 
the field in the first place.  In other words, these individuals may 
decide on a different career path and their discoveries would be left 
undiscovered.  

\section{Conclusions}
Big data is an expanding science in many fields.  The ability to 
digitize, collect, store, and analyze data has never been more than 
it is today.  The type of information that can be used in data analysis 
is expanding every day as well.  Images, videos, and sound are all 
part of the inputs into big data.  Computers are now able to leverage 
natural language processing to make inputs that much easier to collect.  
As this field continues to grow, the ability to leverage it in 
improving healthcare around the world will grow as well.

We are on the edge of a shift in healthcare for the betterment of 
humankind.  Advances will not be limited to one nation or one class of 
people.  While healthcare may not be universal in its application, not 
every person will be able to access the same level of care, there will 
be benefits which can eventually help all people.  A mobile unit which 
can be taken to almost any part of the planet will be able to have the 
knowledge better than most doctors practicing today.  Doctors will have 
access to new drugs, diagnostic information, and treatment plans than 
they ever had before.  They will be able to leverage new advances in 
medicine without having to read as many publications as they can.  They 
will have a tool that reads and learns for them and provides that 
insight on case by case basis.

Through the use of data analysis of sources of data which did not exist 
a decade or so ago, we will be able to identify when a disease is 
starting to spread and react, thus limiting its impact.  Because of 
technology people will be spared from suffering and they will never even 
know it.  By understanding the human genome people who may be more 
susceptible certain diseases can be treated before they take hold.  
Babies will have their genome sequenced while they are still in their 
mother's womb.  This one aspect of the power of big data, the ability 
to process and understand a human genome, may be the single largest 
breakthrough in healthcare.  It can provide insight into how each person 
individually reacts to the world around them and what science can do to 
make that interaction better.  What science can do to help each person 
avoid potential chronic conditions which are not only financially 
costly, but that severely reduce their quality of life or end their 
life.  Through advances in big data we will not only live longer, but 
live better.

\begin{acks}

  The author would like to thank Dr. Gregor von Laszewski for his 
  support throughout this process.  By offering an environment in which 
  students were able to explore areas in big data which interested them, 
  we were all able to further our knowledge individually and collectively.  
  This project is similar to big data itself.  It brought together 
  various thoughts which could be considered data points into the 
  collection of the class.  With access open to all, and potentially 
  future classes, the collection of projects becomes a big data collection 
  unto itself.

\end{acks}

\bibliographystyle{ACM-Reference-Format}
\bibliography{report} 




\end{document}


